For better understanding of Stereo vision few terms are defined in the section below:\\

{\bf Pixel\cite{olea_introduction_2015}:} Pixel is the short form of picture element. It is the physical point in an image. The size of a pixel (in bits) depend on the format of the image. In this thesis the input images are monochrome with pixel sizes of 8 bits.\\

{\bf Resolution\cite{olea_introduction_2015}:} Resolution of an image is number of distinct pixels in each direction. It is specified in WIDTHxHEIGHT format.\\

{\bf Stride\cite{olea_introduction_2015}:} Stride of an image is the number of memory locations between start of successive image lines.\\

The stride of the blender usually depends on the colour mode of the image involved. In the case of this thesis, though the colour mode is monochrome, 121 bits of data are contained for each pixel. Hence colour mode of rgb is chosen with a stride 4 times the x resolution.


\chapter{Appendix}
\label{chap:appendix}

%%%%%%%%%%%%%%%%%%%%%%%%%%%%%%%%%%%%%
%%%%%%%%%%%%%%%%%%%%%%%%%%%%%%%%%%%%%
%%%%%%%%%%%%   SECTION   %%%%%%%%%%%%
%%%%%%%%%%%%%%%%%%%%%%%%%%%%%%%%%%%%%
\section{Nema kernel for morphological erosion}
\label{s:kerero}

\#include "macros.h"

extern "C" \\
\{

  \#define USE\_PIXOUT

  \_\_attribute\_\_((naked)) void shd()
  \{
    unsigned char *input, *output, min;
    int row, column;

    read\_reg(row, "v128.x");//Read constant register value to get the total number of rows
    read\_reg(column, "v128.y");//Read constant register value to get the total number of columns
  
    ui\_vec4 coords;
    read\_coords(coords);//Read coordinates of the thread
    
    read\_reg(input, "v0.w"); /// Read value from interpolator 0
    read\_reg(output, "v0.z"); /// Read value from interpolator 1
  
    if(coords.x>1 \&\& coords.x<(column-1) \&\&coords.y>1 \&\& coords.y<(row-1))
    \{
      min = *input;
      for (int k = -1; k <= 1; k++)
      \{
        for (int l = -1; l <= 1; l++)
        \{
          if (*(input+ k*column+ l) < min)
          \{
            min = *(input + k*column + l);
          \}
        \}
      \}
      *output = min;
    \}
  \}

\} // extern "C"

%%%%%%%%%%%%%%%%%%%%%%%%%%%%%%%%%%%%%
%%%%%%%%%%%%%%%%%%%%%%%%%%%%%%%%%%%%%
%%%%%%%%%%%%   SECTION   %%%%%%%%%%%%
%%%%%%%%%%%%%%%%%%%%%%%%%%%%%%%%%%%%%
%%%%%%%%%%%%%%%%%%%%%%%%%%%%%%%%%%%%%
\section{Section of Nema host program invoking the eGPU}
\label{s:hostero}

...\\
void Nema::erode(unsigned char* input,unsigned char *output, int row, int column)
\{
  int width, height;

  void *bin = nema\_load\_binary("kernels/erode.bin"); /// Load binary
  assert(bin);

  tsi\_cmdl\_add\_cmd(NEMA\_FRAG\_PTR, ::nema\_virt\_to\_phys((uint32\_t)bin));/// Converting virtual address to physical address

  /// Now that the video dimensions are known reprogram rasterizer and the interpolators
  width  = column;
  height = row;

    /// Everything drawn outside this viewport is clipped out
  tsi\_cmdl\_add\_cmd(NEMA\_TARGET\_RESOL,  (height << 16) | width);

    /// Set up rasterizer dimensions
    /// The rasterizer will generate ( width - 1 ) x ( height - 1 ) threads
  tsi\_cmdl\_add\_cmd(NEMA\_VERTEX1\_X, width - 1);
  tsi\_cmdl\_add\_cmd(NEMA\_VERTEX1\_Y, height - 1);

  tsi\_cmdl\_add\_cmd(NEMA\_VAR0\_V1\_0, width);   /// Interpolator 0 stride
  tsi\_cmdl\_add\_cmd(NEMA\_VAR0\_V2\_0, 1);       /// Interpolator 0 step

  tsi\_cmdl\_add\_cmd(NEMA\_VAR0\_V1\_1, width);   /// Interpolator 1 stride
  tsi\_cmdl\_add\_cmd(NEMA\_VAR0\_V2\_1, 1);       /// Interpolator 1 step

  /// Setup Interpolators
  tsi\_cmdl\_add\_cmd(NEMA\_VAR0\_V0\_0, (uint32\_t)nema\_virt\_to\_phys((uint32\_t)input)); /// Interpolator 0 start
  tsi\_cmdl\_add\_cmd(NEMA\_VAR0\_V0\_1, (uint32\_t)nema\_virt\_to\_phys((uint32\_t)output)); /// Interpolator 1 start

  ///Store the image dimensions as constants
  Nema\_setconst(0, 0, row);
  Nema\_setconst(0, 1, column);

  /// Start the rasterizer
  tsi\_cmdl\_add\_cmd(HOLD\_DMA | NEMA\_RAST\_CMD, 0x02);


  nema\_cmdl\_flush\_caches();
  tsi\_cmdl\_emit\_commands(0);
\}\\
...


