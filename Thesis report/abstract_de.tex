\addcontentsline{toc}{chapter}{Resume}
\chapter*{Zusammenfassung}
\label{chap:abstract_de}

%
Die \textit{Zusammenfassung} meiner Arbeit:

Stereo Vision ist ein Verfahren zur 3D-Informationen einer Szene Extrahieren Bilder aus verschiedenen Blickwinkeln aufgenommen werden. Es gibt viele Ans\"atze f\"ur Disparity Map (DM) Sch\"opfung in einem Stereo-Vision-System. Embedded-GPUs sind Low-Power-GPUs in dem gleichen Chip eingebettet zusammen mit der Host-CPU. Stereo Vision ist eine der geeigneten Anwendungen in einem embedded-GPU implementiert werden. In dieser Masterarbeit wird der Effekt von Parametervariationen in einem Stereo-Vision-Algorithmus untersucht. Software-Implementierung des Algorithmus wird durchgef\"uhrt und mehrere Optimierungen sind die Ausf\"uhrungszeit des Algorithmus zu verbessern, durchgef\"uhrt. Optimierungen ergab 31X bis 35X Verbesserung der Ausf\"uhrungszeit in Abh\"angigkeit von der Plattform implementiert. Der Algorithmus wird dann auf Nema eGPU umgesetzt, von Think Silicon Ltd. und deren funktionale Leistung und Ausf\"uhrungszeiten untersucht. Kommentare werden auf der GPU-Architektur auf der Basis von Beobachtungen aufgezeichnet. Zuk\"unftige Arbeiten auf dem Gebiet vorgeschlagen.\\

\noindent
\textsl{\textbf{Stichwort:} Stereo vision, GPU, eGPU, Embedded systems, Parallel programming.}
